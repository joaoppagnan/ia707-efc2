\documentclass[a4paper, 12pt]{article}
\usepackage[utf8]{inputenc}
\usepackage{palatino}
\usepackage[breaklinks=true]{hyperref}
\usepackage{graphicx}
\usepackage{cprotect}
\usepackage{caption}
\usepackage[left=3cm, right=2cm, top=3cm, bottom=2cm]{geometry}
\geometry{a4paper}
\usepackage{fancyhdr}
\usepackage[brazilian]{babel}
\usepackage{siunitx}
\usepackage{subcaption}
\sisetup{detect-all}
\usepackage{float}
\usepackage{ragged2e}
\pagestyle{fancy}
\renewcommand{\headrulewidth}{0pt} 
\lhead{}\chead{}\rhead{}
\lfoot{}\cfoot{\thepage}\rfoot{}
\graphicspath{{figuras/}}
\usepackage{amsfonts}
\usepackage{mathtools}
\usepackage{cleveref}
\usepackage{spverbatim}
\usepackage[framed,numbered,autolinebreaks,useliterate]{mcode}
\setlength{\parskip}{1em}
\usepackage{xspace}
\usepackage{amsmath}
\usepackage{gensymb}
\usepackage{booktabs}
\usepackage{minted}
\usepackage{listings}
\usepackage{bm}
\usepackage{indentfirst}
\usepackage{algorithm}
\usepackage{algpseudocode, lipsum}
\lstset{
    literate={~} {$\sim$}{1}
}
\usepackage[shortlabels]{enumitem}

\newcommand{\MATLAB}{\textsc{Matlab}\xspace}
\newcommand{\SIMULINK}{\textsc{Simulink}\xspace}
\newcommand{\pspice}{\textsc{PSpice}\xspace}
\newcommand{\Python}{\textsc{Python}\xspace}
\newcommand{\tinkercad}{\textsc{TinkerCad}\xspace}
\newcommand{\arduino}{\textsc{Arduino}\xspace}
\newcommand{\sen}{\hspace{2pt}\mathrm{sen}}
\newcommand{\counts}{\textit{counts}\xspace}
\newcommand{\FT}{\text{F.T.}}
\newcommand{\zeros}{\text{Zeros}}
\newcommand{\polos}{\text{Polos}}
\newcommand{\software}{\textit{software}\xspace}
\newcommand{\hardware}{\textit{hardware}\xspace}
\newcommand{\fitness}{\textit{fitness}\xspace}
\newcommand{\fitsha}{\textit{fitness sharing}\xspace}


\newenvironment{brprocess}[1][]
  {\begin{algorithm}[#1]
     \selectlanguage{brazilian}%
     \floatname{algorithm}{Processo}%
     \renewcommand{\algorithmicif}{\textbf{se}}%
     \renewcommand{\algorithmicfor}{\textbf{para}}%
     \renewcommand{\algorithmicdo}{\textbf{faça}}%
     \renewcommand{\algorithmicthen}{\textbf{faça}}%
     \renewcommand{\algorithmicend}{\textbf{fim}}%
     \renewcommand{\algorithmicwhile}{\textbf{enquanto}}%
     \renewcommand{\algorithmicelse}{\textbf{caso contrário}}%
     % Set other language requirements
  }
  {\end{algorithm}}

  \newenvironment{bralgorithm}[1][]
  {\begin{algorithm}[#1]
     \selectlanguage{brazilian}%
     \floatname{algorithm}{Algoritmo}%
     \renewcommand{\algorithmicif}{\textbf{se}}%
     \renewcommand{\algorithmicfor}{\textbf{para}}%
     \renewcommand{\algorithmicdo}{\textbf{faça}}%
     \renewcommand{\algorithmicthen}{\textbf{faça}}%
     \renewcommand{\algorithmicend}{\textbf{fim}}%
     \renewcommand{\algorithmicwhile}{\textbf{enquanto}}%
     \renewcommand{\algorithmicelse}{\textbf{caso contrário}}%
     % Set other language requirements
  }
  {\end{algorithm}}

\begin{document}
\begin{titlepage}
\newcommand{\HRule}{\rule{\linewidth}{0.5mm}}
	
\centering

\includegraphics[width=0.15\textwidth]{logo-unicamp.pdf}\\[0.5cm]	
\textsc{\Large Universidade Estadual de Campinas}\\[2.0cm]
\textsc{\large Faculdade de Engenharia Elétrica e de Computação}\\[0.5cm]
	
\textsc{IA707/EG507 - Computação Evolutiva}\\[2.5cm]
	
{\LARGE \bfseries EFC II}\\[3.5cm]

\begin{minipage}[t]{0.4\textwidth}
	\begin{flushleft}
    \textit{Alunos}\\
    João Pedro O. Pagnan - 199727\\
    João Pedro M. Ferreira - 218762
	\end{flushleft}
\end{minipage}
~
\begin{minipage}[t]{0.4\textwidth}
	\begin{flushright}
		\textit{Professor}\\
		Levy Boccato
	\end{flushright}
\end{minipage}\\[4.5cm]

{Campinas, \today}
	
\vfill\vfill\vfill\vfill\vfill

\includegraphics[width=0.2\textwidth]{logo-feec.png}\\[0.5cm]
\vfill 
	
\end{titlepage}

\justify

Neste exercício, vamos abordar o problema de otimização de função multimodal, com o auxílio de um algoritmo genético (GA, do inglês \textit{genetic algorithm}). Iremos considerar a função contínua
$$f (x, y) = x \sen(4\pi x) - y \sen(4\pi y + \pi) + 1,$$
onde $x, y \in [-1, 2]$ e desejamos determinar o ponto $(x^{*} , y^{*})$ que maximize a função $f(\cdot )$.

\textbf{(a) Proponha um algoritmo genético (AG) para encontrar o máximo global de $f(x, y)$. Descreva todos os elementos que o compõem – codificação, função de \fitness, operadores, etc. — e justifique suas escolhas.}

Para começar a formular o problema deve-se pensar sobre a representação das soluções candidatas e, em seguida, a função \fitness, seguida dos operadores e, por fim, a codificação e parâmetros utilizados para os operadores.

Para este problema foi escolhido utilizar a codificação real, ou seja, um vetor de dois atributos reais da forma
\begin{equation}
    \mathbf{z} = [x, y],
\end{equation} 
sendo $x$ e $y$ as variáveis de $f$.

Esta escolha foi feita pois o problema a ser resolvido se trata de uma otimização de uma função multimodal das variáveis $x$ e $y$.

A função de avaliação que será utilizada é a própria função $f(x, y)$, sendo que deseja-se maximizá-la. 
Desta forma, os valores maiores da função $f(x,y)$ das soluções produzem melhores valores de \textit{fitness} e, consequentemente, a função de avaliação se aproxima da solução ótima do problema.

O algoritmo genético a ser utilizado possui as seguintes etapas:
\begin{enumerate}
    \item Gerar a população inicial com $N$ indivíduos;
    \item Calcular o \textit{fitness} para cada indivíduo da população inicial;
    \item Aplicar o operador de seleção para selecionar dois indivíduos baseado em seus valores de \textit{fitness};
    \item Aplicar o operador de recombinação para gerar dois descendentes;
    \item Aplicar o operador de mutação nos descendentes para que os respectivos genes tenham uma probabilidade $p_m$ de serem mutados;
    \item Caso o número total de indivíduos e descendentes for menor que $2N$, repetir do passo 3 ao 5;
    \item Calcular o \textit{fitness} dos descendentes;
    \item Eliminar os $N$ indivíduos de menor \textit{fitness};
    \item Repetir do passo 3 ao 8 até o critério de parada ser atingido.
\end{enumerate}

A \textbf{geração da população inicial} será da forma indicada no processo \ref{alg:pop-inicial}, sendo $N$ o tamanho da população, $n$ é o número de variáveis da função multimodal, $min$ o valor mínimo do espaço de busca e $max$ o valor máximo do espaço de busca.
\begin{brprocess}[!ht]
\textbf{Paramêtros}: Tamanho da população desejada e tamanho do cromossomo\\
\textbf{Saída}: População gerada
\cprotect\caption{Geração aleatória da população inicial (\verb|gerar_pop(N, n, min, max)|)}
\begin{algorithmic}
\State população $\gets [\;]$
\State $i\gets 0$

\While{$i < N$}
    \State cromossomo $\gets [\;]$
    \State $j \gets 0$
    \While{$j < n$}
        \State gene $\gets$ alelo $\in \{min, max\}$
        \State cromossomo $[j] \gets$ gene
        \State $j \gets j + 1$
    \EndWhile
    \State \textit{fitness} $\gets 0$
    \State população $[i] \gets$ [cromossomo, \textit{fitness}]
    \State $i \gets i + 1$
\EndWhile
\State \textbf{retorna} população
\end{algorithmic}
\label{alg:pop-inicial}
\end{brprocess}

Vale ser mencionado que, para esta representação, composta por $n$ variáveis do problema, a quantidade de alelos é igual ao tamanho do cromossomo. Os alelos são números reais que pertencem ao intervalo $(min,max)$. Além disso, cada indivíduo da população irá armazenar o seu cromossomo e seu valor de \textit{fitness}.

O \textbf{cálculo do \textit{fitness}} seguirá a equação dada do problema e pode ser descrito pelo processo \ref{alg:fitness}.
\begin{brprocess}[!ht]
\textbf{Entrada}: população\\
\textbf{Saída}: população com os valores de \textit{Fitness} atualizado
\cprotect\caption{Cálculo do \textit{fitness} (\verb|calc_fitness(populacao)|)}
\begin{algorithmic}
\For{indivíduo \textbf{em} população}
    \If{indivíduo[1] = 0}
        \State{fitness = f(individuo[0][0],individuo[0][1])}
        \State indivíduo[1] = fitness
    \EndIf
\EndFor
\State \textbf{retorna} população
\end{algorithmic}
\label{alg:fitness}
\end{brprocess}

Na quarta etapa, o GA aplica o \textbf{operador de seleção de torneio} na população. Esse método utiliza o parâmetro $q$, que é a igual a quantidade de indivíduos da população selecionados por torneio. Esse método é executado duas vezes no algoritmo genético, de forma a determinar dois pais para a recombinação e é descrito pelo processo \ref{alg:selecao}.
\begin{brprocess}[!ht]
    \cprotect\caption{Operador de seleção por torneio (\verb|selecao_torneio(populacao,|
    \verb|N, q_torneio)|}
    \textbf{Entrada:} população\\
    \textbf{Parâmetros}: quantidade de indivíduos da população selecionados por torneio e tamanho da população inicial\\
    \textbf{Saída}: o melhor indivíduo escolhido no torneio
    \begin{algorithmic}
            \State $i \gets 0$
            \State indivíduos participantes $\gets [\;]$
            \While{$i < q$}
                \State índice indivíduo selecionado $\gets n \in \{0, 1, ..., N - 1\}$
                \State indivíduos participantes $[i] \gets$ população [índice indivíduo selecionado]
                \State $i \gets i + 1$
            \EndWhile
            \State indice do melhor indivíduo $\gets$ indivíduos participantes[\verb|argmax|(indivíduos participantes [:,1]])
            \State melhor indivíduo $\gets$ indivíduos participantes[índice do melhor indivíduo]
            \State \textbf{retorna} melhor indivíduo
    \end{algorithmic}
    \label{alg:selecao}
\end{brprocess}

Este operador foi escolhido por possibilitar o ajuste da pressão seletiva na população pelo parâmetro $q$.

Em seguida, na quinta etapa, o \textbf{operador de recombinação} que será aplicado foi o método de \textit{crossover aritmético}, nele o vetor (filho) gerado resulta de uma combinação linear (convexa)
de dois vetores (pais). Sendo $P1$ e $P2$ os pais e $D1$ e $D2$ os descendentes, esse método é descrito pelo processo \ref{alg:recombinacao}.

\begin{brprocess}[!ht]
    \cprotect\caption{Operador de recombinação (\verb|crossover_aritmetico(cromossomo_p1,|
    \verb|cromossomo_p2,| \verb|min,| \verb|max|)}
    \textbf{Entrada}: cromossomo dos dois pais, mínimo e máximo do espaço de busca\\
    \textbf{Saída}: cromossomo de ambos os filhos
    \begin{algorithmic}
        \State $i \gets 0$
        \State descendentes $\gets [\;]$
        \While{$i < 2$}
            \State $j \gets 0$
            \State {pesos $\gets [\;]$}
            \While{$j < len(cromossomo\_p1)$}
                \State{pesos[j] $\gets v \in \{0, ..., p_m\, ..., 1\}, v \in \mathbb{R}$}
                \State $j \gets j + 1$ 
            \EndWhile
            \State $k \gets 0$
            \State filho $\gets [\;]$
            \While{$k < len(cromossomo\_p1)$}
                \State {gene1 $\gets cromossomo_\p1[j]$}
                \State {gene2 $\gets cromossomo_\p2[j]$}
                \State {novo-gene $\gets pesos[0]\cdot gene1 + pesos[1]\cdot gene2 $}
                \If{novo-gene$<$ min}
                    \State {novo-gene $\gets$ min}
                \EndIf
                \If{novo-gene$>$ max}
                    \State {novo-gene $\gets$ max}
                \EndIf
                \State{filho[k] $\gets$ novo-gene}
                \State $k \gets k + 1$
            \EndWhile
            \State{descendentes[i] $\gets$ [filho, 0]}
            \State $i \gets i + 1$        
        \EndWhile
        \State \textbf{retorna} descendentes
    \end{algorithmic}
    \label{alg:recombinacao}
\end{brprocess}

Por fim, resta detalhar o \textbf{operador de mutação} utilizado. Neste caso, foi escolhido a mutação uniforme que consiste em pegar uma fração do intervalo de busca que chamaremos de $f_m$, e a partir disso será determinado o intervalo da distribuição, $r = f_m \cdot (max-min)$ assim a distribuição uniforme é dada por $U(-r,r)^l$.

O processo pode ser descrito da forma indicada no processo \ref{alg:mutacao}.
\begin{brprocess}[!ht]
    \cprotect\caption{Operador de mutação (\verb|mutacao_uniforme(cromossomo,|
    \verb|p_mutacao,|\verb|f_mutacao,| \verb|min,| \verb|max|)}
    \textbf{Entrada}: cromossomo a ser mutado e valores mínimo e máximo do espaço de busca\\
    \textbf{Parâmetros}: probabilidade da mutação ocorrer e fator de mutação que vai determinar um intervalo em que o gene pode ser mutado\\
    \textbf{Saída}: cromossomo mutado
    \begin{algorithmic}
        \State $i \gets 0$
        \State{número $\gets k \in \{0, ..., p_m\, ..., 1\}, k \in \mathbb{R}$}
        \For{i \textbf{em} len(cromossomo)}
            \If{indivíduo[1] = 0}
                \State{range-mutacao $\gets (max - min)\cdot f_m$}
                \State {valor-mutacao $\gets r \in \{$-range-mutacao, range-mutacao$\}, r \in \mathbb{R}$}
                \State {cromossomo[i] $\gets$ cromossomo[i] + valor-mutacao}
                \If{cromossomo[i]$<$ min}
                    \State {cromossomo[i] $\gets$ min}
                \EndIf
                \If{cromossomo[i]$>$ max}
                    \State {cromossomo[i] $\gets$ max}
                \EndIf
            \EndIf
        \EndFor
    \end{algorithmic}
    \label{alg:mutacao}
\end{brprocess}

Com isso, foram descritos os operadores escolhidos para serem utilizados na geração de indivíduos desse algoritmo genético.

Estes operadores serão executados nessa sequência até que a população tenha $2N$ soluções, quando ocorrerá uma eliminação dos $N$ indivíduos de menor \textit{fitness}. Esse processo será repetido até o critério de parada ser atingido. 

O \textbf{critério de parada} escolhido para o algoritmo foi o número de gerações $n_g$. 

Esse algoritmo tem, portanto, os seguintes parâmetros:
\begin{itemize}
    \item $N$: tamanho da população inicial;
    \item $q$: quantidade de indivíduos presentes no torneio;
    \item $f_m$: fator da mutação.
    \item $p_m$: probabilidade de ocorrer uma mutação;
    \item $n_g$: número de gerações.
    \item $min$: menor valor do espaço de busca.
    \item $max$: maior valor do espaço de busca.
\end{itemize}

O valor de cada parâmetro foi determinado de maneira exploratória, sendo que os valores escolhidos estão indicados na tabela \ref{tab:valores-parametros}, vale ressaltar que os valores do espaço de busca já foram pré determinado.
\begin{table}[H]
    \centering
    \begin{tabular}{c c}
    \toprule
        $\bm{N}$ & 200\\
        $\bm{q}$ & 20\\
        $\bm{f_m}$ & 0.1\\
        $\bm{p_m}$ & 0.5\\
        $\bm{n_g}$ & 200\\
        $\bm{min}$ & -1\\
        $\bm{max}$ & 2\\
    \bottomrule
    \end{tabular}
    \caption{Valores para os parâmetros do algoritmo genético.}
    \label{tab:valores-parametros}
\end{table}

\textbf{(b) Para cada uma de 5 execuções independentes do AG, apresente as curvas de \fitness médio e de \fitness do melhor indivíduo em função do número de gerações. Mostre também, tendo por pano de fundo as curvas de nível da função $f(x, y)$, a distribuição dos indivíduos da população final em cada execução. Com base nesse conjunto de gráficos, busque analisar o algoritmo em termos de sua eficiência de busca e manutenção de diversidade.}

Executando o algoritmo construído por 5 vezes, e em cada caso gerando uma nova população inicial aleatória, foram obtidas os seguintes gráficos de curva do fitness médio e melhor fitness e também das curvas de nível com a distribuição da população final. Foram divididas em duas figuras \ref{fig:curvas1} e \ref{fig:curvas2} para que a visualização fosse adequada. 

\begin{figure}[!ht]
\centering
\begin{subfigure}{0.8\textwidth}
    \includegraphics[width=\textwidth]{figuras/b1.jpg}
\end{subfigure}
\hfill
\\
\centering
\begin{subfigure}{0.8\textwidth}
    \includegraphics[width=\textwidth]{figuras/b2.jpg}
\end{subfigure}
\hfill
\\
\centering
\begin{subfigure}{0.8\textwidth}
    \includegraphics[width=\textwidth]{figuras/b3.jpg}
\end{subfigure}
\hfill
\caption{Curvas do fitness médio da população e do melhor fitness para cada uma das cinco realizações do algoritmo genético.}
\label{fig:curvas1}
\end{figure}
\begin{figure}[!ht]
\centering
\begin{subfigure}{0.8\textwidth}
    \includegraphics[width=\textwidth]{figuras/b4.jpg}
\end{subfigure}
\hfill
\\
\centering
\begin{subfigure}{0.8\textwidth}
    \includegraphics[width=\textwidth]{figuras/b5.jpg}
\end{subfigure}
\hfill
\caption{Curvas do fitness médio da população e do melhor fitness para cada uma das cinco realizações do algoritmo genético.}
\label{fig:curvas2}
\end{figure}

É interessante notar que, em todas as realizações, o melhor fitness foi sempre bem maior do que o fitness médio na geração 0, ou seja, em relação ao custo médio da população inicial aleatória. Nas gerações seguintes, vê-se que houve um aumento brusco do fitness médio da população, se assemelhando a uma função logarítmica crescente, indicando que os genes do cromossomo do melhor indivíduo da população inicial estão se propagando pelas gerações e, ao cromossomo se recombinar com o cromossomo de outros indivíduos ao longo das gerações do algoritmo, se torna mais parecido com o da solução ótima. Percebe-se que a curva do fitness médio vai chegando mais próximo da curva do melhor fitness, isso acontece pois a diversidade está sendo perdida, já que no fim de cada geração escolhe-se sempre os indivíduos com melhor fitness.

Analisando o gráfico de curvas de nível, é possível notar pela distribuição da população final, que a grande maioria tende a ir pra região de ótimo global, embora ainda existam indivíduos em outros picos próximos ao máximo global, mas é uma quantidade baixa em relação a população inteira. 

Portanto, pode-se concluir que o algoritmo genético desenvolvido teve um bom desempenho em buscar o ótimo global da função analisada, dado que muitos indivíduos se encontram ou no ou em torno do pico mais alto.

\textbf{(c) Implemente agora o método de \fitsha. Comente as escolhas feitas para os valores dos parâmetros (e.g., $\sigma_s$) e de operadores (caso alguma modificação em relação ao item (b) tenha sido necessária). Repita o procedimento do item (b) para analisar o comportamento do \fitsha.}

Nesta parte adicionou-se ao algoritmo genético o \fitsha, que consiste em 
reduzir o fitness do indivíduo i puro por um contador de nichos, que mede a quantidade de indivíduos esse indivíduo i compartilha fitness.
$$\bar{f}(x_i) = \frac{f(x-i)}{c_i}$$
Já $c_i$ é determinado através:
$$c_i = \sum_{j = 0}^{N} sh(d(x_i,x_j))$$

Onde j, são todos os outros indivíduos da população. Por fim, a função de sharing que será utilizada é dada por:
$$sh(d(x_i,x_j)) = \begin{cases}
    1 - (\frac{d(x_i,x_j)}{\sigma_s})^{\aplha_s} &, d(x_i,x_j) < \sigma_s \\
    0 & , \text{caso contrário}
\end{cases}$$

Dado essas informações, é possível adicionar o \fitsha ao algoritmo, entretanto, é necessário explicar que a função para calcular o \fitness se manterá, uma coluna a mais será incluída nos indivíduos para guardar o \fitsha e essa nova função será chamada dentro da seleção por torneio, isso acontecerá, pois ao escolher um indivíduo aleatoriamente para ser o pai da recombinação, será avaliado o \fitsha e dentre os escolhidos para o torneio o vencedor será o indivíduo de maior \fitsha. Assim, a população será diversificada, sendo escolhidos indivíduos de bons fitness de regiões promissoras. O processo pode ser descrito da forma indicada no processo \ref{alg:fitness_share}.
\begin{brprocess}[!ht]
    \cprotect\caption{Operador de \fitsha (\verb|fitness_share(individuo,|
    \verb|populacao,|\sigma_s, \alpha_s)}
    \textbf{Entrada}: individuo a ser calculado o fitness sharing e a população que será utilizado para tal cálculo\\
    \textbf{Parâmetros}: $\sigma_s$ que determina o limiar de similaridade e $\aplha_s$ que denota a forma denota a forma da função de share\\
    \textbf{Saída}: indivíduo com \fitsha calculado
    \begin{algorithmic}
        \State sharing $\gets [\;]$
        \State $i \gets 0$
        \For{i \textbf{em} len(populacao)}            
            \State{d $\gets$ distancia(individuo[0],populacao[i][0]}
            \If{$d < \sigma$}
                \State {resultado $\gets 1 - (\frac{d(x_i,x_j)}{\sigma_s})^{\aplha_s}$}                
                \State {sharing $\gets$ resultado}
            \Else
                \State {sharing[i] $\gets$ 0}
            \EndIf       
        \EndFor        
        \State fitness-sh $\gets \frac{individuo[1]}{sum(sharing)}$
        \State {individuo[2] $\gets$ fitness-sh} 
        \State \textbf{retorna} individuo
    \end{algorithmic}
    \label{alg:fitness_share}
\end{brprocess}

Essa nova função tem a mais, os seguintes parâmetros:
\begin{itemize}
    \item $\sigma_s$: limiar de similaridade;
    \item $\alpha_s$: regulação da forma da função
\end{itemize}

O valor de cada parâmetro foi determinado de maneira exploratória, sendo que os valores escolhidos estão indicados na tabela \ref{tab:valores-parametros2}, vale ressaltar que os valores do espaço de busca já foram pré determinado, e os outros parâmetros continuaram iguais, para fim de comparação entre o uso e sem o uso do método de \fitsha.
\begin{table}[H]
    \centering
    \begin{tabular}{c c}
    \toprule
        $\bm{\sigma}$ & 0.4\\
        $\bm{\alpha}$ & 1\\
    \bottomrule
    \end{tabular}
    \caption{Valores dos parâmetros do \fitsha.}
    \label{tab:valores-parametros2}
\end{table}

Executando o algoritmo modificado por 5 vezes, e em cada caso gerando uma nova população inicial aleatória, foram obtidas os seguintes gráficos de curva do fitness médio e melhor fitness e também das curvas de nível com a distribuição da população final. Foram divididas em duas figuras \ref{fig:curvas3} e \ref{fig:curvas4} para que a visualização fosse adequada. 

\begin{figure}[!ht]
\centering
\begin{subfigure}{0.8\textwidth}
    \includegraphics[width=\textwidth]{figuras/c1.jpg}
\end{subfigure}
\hfill
\\
\centering
\begin{subfigure}{0.8\textwidth}
    \includegraphics[width=\textwidth]{figuras/c2.jpg}
\end{subfigure}
\hfill
\\
\centering
\begin{subfigure}{0.8\textwidth}
    \includegraphics[width=\textwidth]{figuras/c3.jpg}
\end{subfigure}
\hfill
\caption{Curvas do fitness médio da população e do melhor fitness para cada uma das cinco realizações do algoritmo genético com \fitsha.}
\label{fig:curvas3}
\end{figure}
\begin{figure}[!ht]
\centering
\begin{subfigure}{0.8\textwidth}
    \includegraphics[width=\textwidth]{figuras/c4.jpg}
\end{subfigure}
\hfill
\\
\centering
\begin{subfigure}{0.8\textwidth}
    \includegraphics[width=\textwidth]{figuras/c5.jpg}
\end{subfigure}
\hfill
\caption{Curvas do fitness médio da população e do melhor fitness para cada uma das cinco realizações do algoritmo genético com \fitsha.}
\label{fig:curvas4}
\end{figure}

 Para esse caso, pode-se notar que, em todas as realizações, o melhor fitness foi sempre maior do que o fitness médio na geração 0, ou seja, em relação ao custo médio da população inicial aleatória. Nas gerações seguintes, vê-se que houve um aumento brusco do fitness médio da população, se assemelhando a uma função logarítmica crescente, indicando que os genes do cromossomo do melhor indivíduo da população inicial estão se propagando pelas gerações e, ao cromossomo se recombinar com o cromossomo de outros indivíduos ao longo das gerações do algoritmo, se torna mais parecido com o da solução ótima. A diferença para o primeiro algoritmo se percebe pela maior distância entre curva do fitness médio e a curva do melhor fitness, ainda que estejam próximas, elas estão mais afastadas do que o algoritmo sem o uso de \fitsha.

A questão da maior diversidade fica mais clara quando visualiza-se o gráfico das curvas de nível, é possível notar pela distribuição da população final, que existem mais pontos onde os indivíduos se juntara. Nota-se que existem oito picos com alguma quantidade populacional. 

Portanto, pode-se concluir que o algoritmo genético com \fitsha teve um bom desempenho em diversificar a população em mais pontos do que o algoritmo anterior.

\textbf{(d) Introduza, por fim, o esquema de restrição de cruzamento (seguindo o espírito da abordagem de especiação) proposto por \cite{deb1989genetic} Analise o impacto da inserção deste mecanismo na evolução da população e, em última análise, no desempenho do algoritmo.}

Nesta parte adicionou-se ao algoritmo genético o método de especiação, que tende a restringir a reprodução de indivíduos em uma determinada região, fazendo com que os indivíduos se encontre em picos promissores da superfície. Para que isso aconteça, limita-se a região de reprodução através do parâmetro $\sigma_m$. Essa função consiste em escolher pares de pais próximos, caso estejam distantes a seleção do segundo pai vai continuar até que encontre pais dentro desse limiar

Dado essas informações, é possível adicionar a restrição de cruzamento, apenas alterando a maneira de selecionar os pais, isto é, quando encontrar dois pais próximos eles serão utilizados para a recombinação, portanto não será alterado a maneira nem de seleção e nem da recombinação. O processo pode ser descrito da forma indicada no processo \ref{alg:especiacao}.
\begin{brprocess}[!ht]
    \cprotect\caption{Operador de especiação (\verb|especiação(populacao,|
    \verb|n_populacao,|\verb|q_torneio,|\sigma_m)}
    \textbf{Entrada}: população e o tamanho da população\\
    \textbf{Parâmetros}: $\sigma_m$ que determina o limiar de proximidade dos pais e quantidade de indivíduos da população selecionados por torneio\\
    \textbf{Saída}: pais selecionados para posteriormente recombinação
    \begin{algorithmic}
        \State pais $\gets [\;]$
        \State $i \gets 0$
        \State pai1 $\gets $ torneio-fsh(populacao,n_populacao,q_torneio)
        \State {pais[0] $\gets$ pai1} 
        \While{i != 1)}            
            \State pai2 $\gets $ torneio-fsh(populacao,n_populacao,q_torneio)          
            \State{d $\gets$ distancia(pai1[0],pai2[i][0]}
            \If{$d < \sigma_m$}
                \State {pais[0] $\gets$ pai1}
                \State {i $\gets$ 1}
            \EndIf       
        \EndWhile         
        \State \textbf{retorna} pais
    \end{algorithmic}
    \label{alg:especiacao}
\end{brprocess}

Essa nova função tem a mais, o seguinte parâmetro:
\begin{itemize}
    \item $\sigma_m$: limiar de proximidade;
\end{itemize}

O valor do parâmetro de limiar de proximidade foi determinado de maneira exploratória onde $\sigma_m = 0.$, esperava-se que dentro das regiões que cada espécie se encontrou, os indivíduos se encontrassem mais aglomerados nos picos do que aquilo que será notado nos gráficos.

Executando o algoritmo modificado por 5 vezes, e em cada caso gerando uma nova população inicial aleatória, foram obtidas os seguintes gráficos de curva do fitness médio e melhor fitness e também das curvas de nível com a distribuição da população final. Foram divididas em duas figuras \ref{fig:curvas5} e \ref{fig:curvas6} para que a visualização fosse adequada. 

\begin{figure}[!ht]
\centering
\begin{subfigure}{0.8\textwidth}
    \includegraphics[width=\textwidth]{figuras/d1.jpg}
\end{subfigure}
\hfill
\\
\centering
\begin{subfigure}{0.8\textwidth}
    \includegraphics[width=\textwidth]{figuras/d2.jpg}
\end{subfigure}
\hfill
\\
\centering
\begin{subfigure}{0.8\textwidth}
    \includegraphics[width=\textwidth]{figuras/d3.jpg}
\end{subfigure}
\hfill
\caption{Curvas do fitness médio da população e do melhor fitness para cada uma das cinco realizações do algoritmo genético com especiação.}
\label{fig:curvas5}
\end{figure}
\begin{figure}[!ht]
\centering
\begin{subfigure}{0.8\textwidth}
    \includegraphics[width=\textwidth]{figuras/d4.jpg}
\end{subfigure}
\hfill
\\
\centering
\begin{subfigure}{0.8\textwidth}
    \includegraphics[width=\textwidth]{figuras/d5.jpg}
\end{subfigure}
\hfill
\caption{Curvas do fitness médio da população e do melhor fitness para cada uma das cinco realizações do algoritmo genético com especiação.}
\label{fig:curvas6}
\end{figure}

Como já foi dito ao explicar o parâmetro $\sigma_m$, os valores testados, não geraram bons resultados, visto que a ideia era que os indivíduos das regiões estabelecidas pelo algoritmo, se concentrassem mais nos picos das mesmas.

Sendo crítico, esse método deveria trazer mais diversidade para o modelo, mas como se pode observar, não teve grandes melhoras em comparação ao método de \fitsha sozinho. Não sei se o conceito foi bem interpretado ou se a implementação não foi adequada, mas conclui-se que para essa parte o algoritmo não desempenhou o que se esperava.

\textbf{Observação:} Nos gráficos do algoritmo com especiação não contam com os títulos referentes a cada iteração, pois dessa vez não foi rodado em um loop e sim individualmente, pois em loop estava levando muito tempo e acabou travando no meio, para mitigar esse problema, rodando individualmente, se travasse em algum, era só reiniciar e rodar a quantidade que faltasse.

\clearpage

\bibliographystyle{ieeetr}

\bibliography{bib}

\pdfinfo{
   /Title  (IA707 - EFC II - 199727 e 218762)
}
 \end{document}